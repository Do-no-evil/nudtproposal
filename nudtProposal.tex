%%%%% --------------------------------------------------------------------------------
%%
%%                           Document Template of NUDT proposal
%%
%%%%% --------------------------------------------------------------------------------
%% Copyright (C) Hanlin Tan <hanlin_tan@nudt.edu.cn> 
%% This is free software: you can redistribute it and/or modify it
%% under the terms of the GNU General Public License as published by
%% the Free Software Foundation, either version 3 of the License, or
%% (at your option) any later version.
%%%%% --------------------------------------------------------------------------------
%% Last Updated: 2017.01.06
%%%%************************ Document Class Declaration ******************************
%%
%\documentclass{ctexart}

\documentclass{Style/nudtproposal}% thesis template of UCAS
%% Multiple optional arguments:
%% [scheme = plain] % for thesis writing of international students
%% [<singlesided|doublesided|printcopy>] % single-sided, double-sided, or print layout
%% [draftversion] % show draft version information, default is no show
%% [fontset = <adobe|...>] % specify font set, default is automatic detection
%% [standard options for ctex class]
%%%%% --------------------------------------------------------------------------------
%%
%%%%************************* Command Define and Settings ****************************
%%
\usepackage{Style/commons}% common settings
%% usage: \usepackage[option1,option2,...,optionN]{commons}
%% Multiple optional arguments:
%% [myhdr] % one available header and footer style, will enable fancyhdr
%% [lscape] % provide landscape layout environment
%% [geometry] % configure page layout by geometry package
%% [list] % enable enhanced list environments, useful for Algorithm and Coding
%% [color] % enable color package to use color, default package is xcolor
%% [background] % enable page background, will auto enable color package
%% [tikz] % enable tikz for complex diagrams, will auto enbale color package
%% [table] % enable a table package for complex tables, default is ctable
%% [math] % enable some extra math packages
\usepackage{Style/custom}% user defined commands
\usepackage[backend=biber, style=Biblio/nudtcaspervector,utf8, sorting=none]{biblatex} % 设定引用格式
\addbibresource{Biblio/ref.bib}
\usepackage{Style/nudtstyle} % 包含作者自定义的格式和命令
% 设置参考文献文件
\addbibresource{Biblio/ref.bib}
%%%%% ---------------------------------------------------------------------------------
%%%%% ---------------警告:以上内容请勿随意修改,除非你清楚自己在做什么------------


%%%%% --------------提示:修改本节内容用于设置文档,请仔细阅读---------------------
%% 
%% 编译环境:texlive-2015。
%% 推荐IDE:texstudio(WinXXX 太挫了)。
%% 编译选项:tex编译器选择xelatex, 参考文献编译器选择biber(不能用bibtex)!
%% 以上环境配置经过作者测试,确定可以正常使用。

%%%%% ---------------提示:本参数提供一个参考文献格式BUG的临时解决方案------------
% 是否将参考文献放入表格。这个选项的设置因为参考文献格式有一个暂时无解的BUG。设置为yes之前之前,必须先
% 设置为no编译一次。否则正文中引用数字都是0。
% 最终提交前将此参数改为yes编译一次。注意:设置为yes后编译第二次就会出现正文中引用数字都是0的BUG!
% 如果设置为yes后,还需要修改正文,那么改为no,编译  两次 参考文献才会正常!
\enabletablebib{no}   % 最终提交前将此参数改为yes编译一次! 如需修改正文,改为no后编译两次参考文献才会正常!   

%% 以下参数用于设置文档首页和页眉信息
\proposaltype{doctor}          % 研究生类别:硕士设置为master,博士设置为doctor 
\enabletableofcontents{no}   % 是否生成目录:如果需要目录设置为yes,否则设置为no。我校开题报告默认没有目录
\proposalnumber{\underline{\hbox to 10mm{}}}          % 编号:默认是下划线,如果你知道编号,设为真实编号
\classification{公开}              % 密级:公开,秘密,机密或者绝密
\nudttitle{国防科学技术大学开题报告}{\LaTeX{} 模板} % 因title一般都很长需要两行,第一参数为第一行内容,第二个参数为第二行内容
\author{谭同学}                    % 作者
\authorid{160590xx}            % 学号
\advisor{张老师}                   % 导师
\advisortitle{教~~~~授}         % 职称
\degreetype{工学}                 % 学位类别
\major{控制科学与工程}          % 一级学科
\field{图像处理}                      % 研究方向
\institute{信息系统与管理学院}% 学院
\chinesedate{2017~年~03~月~01日} % 开题日期
\formdate{二零一八年一月}     % 制表月份 注意:用“〇”可能会出现字体不显示的问题,所以这里改为了“零”

%% 在设置完以上参数后,修改Tex文件下对应文件以完成开题报告。
%%%%% ---------------------------------------------------------------


%%%%% ---------------警告:以下内容请勿随意修改,除非你清楚自己在做什么------------

%%%%******************************** Content *****************************************
%%
\begin{document}
%%
%%%%% --------------------------------------------------------------------------------
%%
%%%%******************************** Frontmatter *************************************
%%
\pagenumbering{roman}% restart page numbers with arabic style
%%% Generate Title
%%
\maketitle

%%%%% --------------------------------------------------------------------------------
%%
%%%%******************************** Mainmatter **************************************
%%

%% 添加正文内容
\pagenumbering{arabic}% restart page numbers with arabic style
\mdfsetup{skipabove=0pt,skipbelow=0pt}
%% 包含正文各个章节,请编辑章节文件修改相应的内容
%%%%% --------------------------------------------------------------------------------
%%
%%%%******************************* Main Content *************************************
%%
%%% ++++++++++++++++++++++++++++++++++++++++++++++++++++++++++++++++++++++++++++++++++




\section{学位论文选题的立论依据}


\begin{mdframed}[everyline=true]
	
\subsection{课题来源}
自拟。


\subsection{基本概念}
\subsubsection{异常事件}
异常,新华词典的解释是“不同于平常”\upcite{a6-1}。从分类的角度看,异常与正常是两个大类别,异常内部又可以分成打架、撞车等小类别。从概率的角度看\upcite{1-4},“平常”是大多数,而“异常”就是少数,所以异常事件,则可解释为“小概率事件”。
    异常事件的分类有很多角度。根据场景运动目标的多少,可以分为拥挤场景的异常事件和不拥挤场景的异常事件。这种分类主要根据基于跟踪和轨迹分析的异常检测方法能否适用。拥挤场景现有的跟踪方法都会失效,而不拥挤场景基于跟踪和轨迹分析的方法是可能奏效的。根据异常事件的规模,可以分为全局异常事件(如图 1)和局部异常事件(如图 2)。这种分类可以用于决定异常警报的级别。根据异常事件是基于先验知识还是场景学习,可以分为特定类型异常事件和广义异常事件\upcite{biblatex}。

%% ----------------注意: mdframed 中不能使用 figure 和 table。用以下示例方法替代。----------------
\begin{center}
	\includegraphics[width=0.7\linewidth]{Img/fig1}
	\captionof{figure}{人群四散逃离的异常事件(全局异常事件)} % \caption{} 改为 \captionof{figure}{}
	\label{fig:fig1}
\end{center}

\begin{center}
	\includegraphics[width=0.7\linewidth]{Img/fig2}
	\captionof{figure}{摩托车违章逆行的异常事件(局部异常事件)} % \caption{} 改为 \captionof{figure}{}
	\label{fig:fig2}
\end{center}


\begin{center}
	\captionof{table}{一张表}   % \caption{} 改为 \captionof{table}{}
	\begin{tabular}{lrrr} \hline
		年份        & 乡村 & 城市 & 所有   \\ \hline
		1983        & 38.7  & 55.6  & 44.7  \\
		1993–1994   & 50.3  & 66.4  & 54.3  \\
		2004–2005   & 50.2  & 69.3  & 55    \\
		2009–2010   & 51.7  & 71.6  & 57.1  \\ \hline
		\multicolumn{4}{@{}l@{}}{\footnotesize 来源: http://tomheaven.cn} 
	\end{tabular}
    \label{tab:tab1}
\end{center}

\subsubsection{广义异常事件}

本课题认为从分类的角度检测到的是特定类型异常事件\upcite{2-5_2},而从概率的角度检测到的是广义异常事件\upcite{li2000}。例如打架斗殴、人群逃散、交通事故都是根据人们的先验知识确定的异常事件,在绝大多数场景中,只要发生这样的事件,就肯定是异常事件。而广义异常事件与特定类型异常事件相对,是指不能由人们的先验知识预先设定类别,而是由监控视频场景决定的异常事件。发生概率低和与场景相关是广义异常事件的本质特征。

例如图 2的摩托车逆行,只有发生在此场景的城市道路上,才是异常事件。如果发生在了无人烟的乡村土路上并不算是异常。而摩托车是不是逆行,也只有放在此特定的场景中才能判断。

\subsection{研究意义}
    随着视频监控在商场、银行、小区、道路等公共场所的广泛部署\upcite{Cong2011Sparse},监控视频数据大量产生。目前监控视频主要还是用于威慑犯罪和事后调取,但视频智能分析的需要一直存在。近期发生了一些引起公众关注的事件再次体现了监控视频异常检测需求的迫切性。IBM深圳公司的一名女经理在地铁口突发心脏病跌倒,虽然正对着监控,却因为监控无人查看而耽误了抢救时间,最终不幸去世。对于这种紧急情况,仅有八分钟的黄金抢救时间,不能及时发现险情和施救生命就会逝去。
    监控视频的异常检测在安防领域、交通管理、城市管理方面有广阔的应用前景。从监控视频中自动发打架斗殴、交通违章、交通事故、人群聚集等事件具有及时发现事故险情,提前发现安全隐患的作用。例如图 3中的行人违规横穿马路,说明此路段存在交通安全隐患,有必要派出交警或者增设警示标志。如果这种情况持续发生,可以考虑架设人行天桥来引导行人。而这种安全隐患靠人工是很难发现和统计的。

  随着计算能力的不断进步,满足视频智能分析需求的计算成本在不断降低,视频智能分析的技术也在不断进步,为监控视频智能分析的普及准备着技术条件,智能监控的时代正在迫近。异常事件检测,作为视频智能分析的重要一环,能够帮助及早发现安全隐患,对异常事件实时发出警报,对于利用监控视频保障安全、处置险情,有重要作用。 
\\[8 cm]
\end{mdframed}

%%% ++++++++++++++++++++++++++++++++++++++++++++++++++++++++++++++++++++++++++++++++++
%   \include ?= \input + \clearpage
\clearpage
\input{Tex/2_referenceSummary.tex}%
\clearpage
%%%%% --------------------------------------------------------------------------------
%%
%%%%******************************* Main Content *************************************
%%
%%% ++++++++++++++++++++++++++++++++++++++++++++++++++++++++++++++++++++++++++++++++++




\section{研究内容}
\begin{mdframed}[everyline=true]

\subsection{研究目标}

\subsection{主要研究内容及拟解决的相关科学问题和技术问题}
	
\subsection{拟采取的研究方法、技术路线、实施方案及可行性分析}

\subsection{预期创新点}

很多内容……
\\[10 cm]
\end{mdframed}


%%% ++++++++++++++++++++++++++++++++++++++++++++++++++++++++++++++++++++++++++++++++++
%
\clearpage
\input{Tex/4_researchCondition.tex}%
\clearpage
%%%%% --------------------------------------------------------------------------------
%%
%%%%******************************* Main Content *************************************
%%
%%% ++++++++++++++++++++++++++++++++++++++++++++++++++++++++++++++++++++++++++++++++++




\section{学位论文工作计划}
{
\noindent
\begin{tabular*}{0.999\textwidth}{| p{0.07\textwidth } <{\centering} | p{0.45\textwidth}  | p{0.162\textwidth} | p{0.20\textwidth}  |}

	\hline 
	\multicolumn{1}{|c|}{起讫日期} & 	\multicolumn{1}{c}{主要完成研究内容} & 	\multicolumn{1}{|c|}{预期成果} \\
	\hline 
	\tabincell{c}{2017年09月 -- \\2018年03月}   &  基础知识学习 &   完成文献搜集与该方向基本知识储备 \\ 
	\hline 
	\tabincell{c}{2018年04月 -- \\2018年06月} &  研究点1 &   完成实验 \\ 
	\hline 
	\tabincell{c}{2018年07月 -- \\2018年08月} &  研究点1 &   发表论文SCI一篇 \\ 
	\hline 
	\tabincell{c}{2018年09月 -- \\2018年10月} &  研究点2 &   完成实验 \\ 
	\hline 
    \tabincell{c}{2018年11月 -- \\2018年12月} &  研究点2 &   发表论文EI一篇 \\ 
    \hline 
    \tabincell{c}{2019年01月 -- \\2019年02月} &  研究点3 &   完成实验 \\ 
    \hline 
    \tabincell{c}{2019年03月 -- \\2019年04月} &  研究点3 &   发表论文EI一篇 \\ 
    \hline 
    \tabincell{c}{2019年05月 -- \\2019年06月} &  研究点4 &   完成实验 \\ 
    \hline 
    \tabincell{c}{2019年07月 -- \\2019年08月} &  研究点4 &   发表论文EI一篇 \\ 
    \hline 
    \tabincell{c}{2019年09月 -- \\2019年09月} &  研究点5 &   完成实验 \\ 
    \hline 
    \tabincell{c}{2019年10月 -- \\2019年10月} &  研究点5 &   发表论文EI一篇 \\ 
    \hline 
	\tabincell{c}{2019年11月 -- \\2020年01月} &  撰写毕业论文 &  完成毕业论文 \\ 
	\hline 
\end{tabular*} 
\\[1 cm]
{\songti 注:每个子阶段不得超过3个月;预期成果中必须包含成果的形式、数量、质量等可考性指标该计划将作为
论文研究进展检查的依据。}
\indent
}


%%% ++++++++++++++++++++++++++++++++++++++++++++++++++++++++++++++++++++++++++++++++++
%
\clearpage
\input{Tex/6_references.tex}%
\clearpage
%%%%% --------------------------------------------------------------------------------
%%
%%%%******************************* Main Content *************************************
%%
%%% ++++++++++++++++++++++++++++++++++++++++++++++++++++++++++++++++++++++++++++++++++




\section{指导教师对开题报告的评语}
\begin{mdframed}[everyline=true]
\indent {\songti{(对1-6项逐项予以评价,并着重对国内/外研究现状的了解情况、研究内容的创新性等方面进行评价,
最终给出是否满足博士/硕士层次学位论文研究要求的综合评价意见)}}

课题评价。……

符合博士研究生开题要求。
\\[12 cm]

\begin{flushright}
    \songti{导师签字:}\ \ \ \ \ \ \ \ \ \ \ \ \ \ \ \ \ \ \ \ \ \ \ \ \ \ \ \ \ \ 
    \vspace{10 mm}

    \ \ \ \ \ \ \ \ \ \ \ 年 \ \ \ \ \ \ 月 \ \ \ \ \ \ \ 日
    \vspace{30 mm}
\end{flushright}

\end{mdframed}

%%% ++++++++++++++++++++++++++++++++++++++++++++++++++++++++++++++++++++++++++++++++++
%
\clearpage
\input{Tex/8_groupComments.tex}%
\clearpage

\end{document}
%%%%% --------------------------------------------------------------------------------
